%%%%%%%%%%%%%%%%%%%%%%%%%%%%%%%%%%%%%%%%%%%%%%%%%%%%%%%%%%%%%%%%%%%%%%%%%%%
%%%%%%%%%%%%%%%%%%%%%% Klausur-Logo (Deckblatt) %%%%%%%%%%%%%%%%%%%%%%%%%%%
%%%%%%%%%%%%%%%%%%%%%%%%%%%%%%%%%%%%%%%%%%%%%%%%%%%%%%%%%%%%%%%%%%%%%%%%%%%
\newcommand{\klogo}{ \sffamily \vspace*{-10mm}
  \begin{minipage}{2cm}
    \includegraphics[scale=0.3]{Bilder/KIT.jpg}
  \end{minipage}
  \hfill
  \begin{minipage}{10.5cm}
    %\textsf{{\Large U}\large{NIVERSIT"AT} \Large{K}\large{ARLSRUHE}} \\
    \textsf{\large{Fakult"at f"ur Informatik}} \\[0.2ex]
    \textsf{\large{Dr. Eunah Cho}}\\  
    \textsf{Institut f�r Anthropomatik (IFA)}    
  \end{minipage}
  \rmfamily
  \vspace*{2cm}

  \centerline{\Huge Tutorial} 
  \vspace{0.3cm}
  \centerline{\Large  zur Klausur \glqq Kognitive Systeme\grqq}
  \vspace{0.3cm}
  \centerline{\Large am 18. September 2012, 11:00 - 12:00 Uhr}
  \vspace{1cm} 
%
  \begin{itemize}
  \item Beschriften Sie bitte gleich zu Beginn jedes L�sungsblatt deutlich
    lesbar mit Ihrem Namen und Ihrer \mbox{Matrikel\-nummer}.
  \item Diese Aufgabenbl�tter werden nicht abgegeben. Tragen Sie Ihre
    L�sung deshalb ausschlie�lich in die f�r jede Aufgabe vorgesehenen
    Bereiche der \underline{L�sungsbl�tter} ein. 
    % L�sungen auf separat abgegebenen Bl�ttern werden nicht gewertet.
  \item Au�er Schreibmaterial, Lineal, Geodreieck und Zirkel sind w�hrend der Klausur keine Hilfsmittel
    zugelassen.  T�uschungsversuche durch Verwendung unzul�ssiger
    Hilfsmittel f�hren unmittelbar zum Ausschluss von der Klausur und zur
    Note \glqq nicht bestanden\grqq. 
%  \item Soweit in der Aufgabenstellung nichts anderes angegeben ist, tragen
%    Sie in die L�sungsbl�tter bitte nur die Endergebnisse ein. Die
%    R�ckseiten der Aufgabenbl�tter k�nnen Sie als \mbox{Konzeptpapier}
%    verwenden. Weiteres Konzeptpapier k�nnen Sie auf Anfrage w�hrend der
%    Klausur erhalten.
  \item Halten Sie Begr�ndungen oder Erkl�rungen bitte so kurz wie m�glich.
    (Der auf den L�sungsbl�ttern f�r eine Aufgabe vorgesehene Platz steht
    �brigens in keinem Zusammenhang mit dem Umfang einer korrekten L�sung!)
  \item Die Gesamtpunktzahl betr�gt 60 Punkte (+ max. 6 Punkte online). Zum Bestehen der Klausur
    sind mindestens 20 Punkte aus der Klausur zu erreichen.
  \item Notieren Sie bitte einen Hinweis auf das Deckblatt der L�sungsbl�tter, falls Sie Ihre Onlinepunkte nicht im Sommersemester 2012 gesammelt haben. Geben Sie hierbei das Semester an, in welchem Sie die Onlinepunkte gesammelt haben. 
  \item Verwenden Sie weder Rotstift noch Bleistift. Alle L\"osungen in rot oder mit Bleistift werden nicht gewertet.
  \end{itemize}
  \vspace{2cm}
  \centerline{\large \emph{\textbf{Viel Erfolg!}}}
  \newpage
  } % \end Klausur-Logo
% 
%%%%%%%%%%%%%%%%%%%%%%%%% Headings (Klausur) %%%%%%%%%%%%%%%%%%%%%%%%%%%%%
\newcommand{\khead}{ \pagestyle{myheadings}
  \markright{\underline{\textsf{Aufgaben zur Klausur \glqq Kognitive Systeme\grqq\  am 18.~September 2012}}}}

%%%%%%%%%%%%%%%%%%%%%%%  Aufgaben-Counter %%%%%%%%%%%%%%%%%%%%%%%%%%%%%%%%
\newcounter{aufgabe} \setcounter{aufgabe}{1}
\newcommand{\aufg}[2]{{\Large\textbf{Tutorial \theaufgabe} \quad \emph{#1}
    \hfill (#2 Punkte) \stepcounter{aufgabe}}}

%%%%%%%%%%%%%%%%%%%%%%%%%%%%%%%%%%%%%%%%%%%%%%%%%%%%%%%%%%%%%%%%%%%%%%%%%%%
%%%%%%%%%%%%%%%%%%%%% L�sungs-Logo %%%%%%%%%%%%%%%%%%%%%%%%%%%%%%%%%%%%%%%%
%%%%%%%%%%%%%%%%%%%%%%%%%%%%%%%%%%%%%%%%%%%%%%%%%%%%%%%%%%%%%%%%%%%%%%%%%%%
\newcommand{\llogo}{ \sffamily \vspace*{-10mm}
  \begin{minipage}{2cm}
    \includegraphics[scale=0.3]{Bilder/KIT.jpg}
  \end{minipage}
  \hfill
  \begin{minipage}{10.5cm}
    %\textsf{{\Large U}\large{NIVERSIT"AT} \Large{K}\large{ARLSRUHE}} \\
    \textsf{\large{Fakult"at f"ur Informatik}} \\[0.2ex]
    \textsf{\large{Prof. Dr. A. Waibel}}\\  
    \textsf{\large{Prof. Dr. R. Dillmann}}\\[0.2ex]  
    \textsf{Institut f�r Anthropomatik (IFA)}    
  \end{minipage}
  \rmfamily
  \vspace*{2cm}

  \centerline{\Huge L�sungsbl�tter}
  \centerline{\Large zur Klausur \glqq Kognitive Systeme\grqq } 
  \vspace{1ex}
  \centerline{\Large am  18. September 2012,  11:00 - 12:00 Uhr}
%
  \begin{table}[htb]
    \begin{center}
      \leavevmode
      \begin{tabular}[t]{|p{5cm}|p{5cm}|p{5cm}|} \hline
        Name: & Vorname: & Matrikelnummer:\\ & & \\  & & \\ \hline
      \end{tabular}
    \end{center}
  \end{table}
  \vspace*{-1cm}
  \renewcommand{\arraystretch}{1.3}
  \begin{table}[!hb]
    \begin{center}
      \leavevmode
      \begin{tabular}[t]{|l|rl|} \hline
        Tutorial 2 \hspace{5.5cm} & \hspace{3.5cm} & von 10 Punkten\\ \hline
        Aufgabe 2 \hspace{5.5cm} & \hspace{3.5cm} & von 10 Punkten\\ \hline 
        Aufgabe 3 \hspace{5.5cm} & \hspace{3.5cm} & von 10 Punkten\\ \hline
        Aufgabe 4 \hspace{5.5cm} & \hspace{3.5cm} & von 10 Punkten\\ \hline 
        Aufgabe 5 \hspace{5.5cm} & \hspace{3.5cm} & von 10 Punkten\\ \hline
        Aufgabe 6 \hspace{5.5cm} & \hspace{3.5cm} & von 10 Punkten\\ \hline        
        Online~~~ \hspace{5.5cm} & \hspace{3.5cm} & von 6 Punkten\\ \hline  
        \hline
        \multicolumn{3}{c}{ }\\ \hline
        \textbf{Gesamtpunktzahl:} &  & \\ \hline
        \multicolumn{3}{c}{ } \\ \hline
        & \multicolumn{2}{|l|}{\textbf{Note:}}\\\hline
      \end{tabular}
    \end{center}
  \end{table}
  \newpage
  } %\end L�sungs-Logo

%%%%%%%%%%%%%%%%%%%%%%%%% Headings (Klausur) %%%%%%%%%%%%%%%%%%%%%%%%%%%%%
\newcommand{\lhead}{ \pagestyle{myheadings}
  \markright{\textsf{Name: \hspace*{4cm} Vorname:\hspace*{3cm}
        Matr.-Nr.: \hspace*{3cm}}}}
%%%%%%%%%%%%%%%%%%%%%%%%%% Aufgaben-Counter %%%%%%%%%%%%%%%%%%%%%%%%%%%%%
\newcounter{loesung} \setcounter{loesung}{1} \newcommand{\loes}{{\Large
    \textbf{Tutorial NMT}} \stepcounter{loesung}}
%%%%%%%%%%%%%%%%%%%%%%%%%%%%%%%%%%%%%%%%%%%%%%%%%%%%%%%%%%%%%%%%%%%%%%%%

%%%%%%%%%%%%%%%%%%%%%%%%%%%%%%%%%%%%%%%%%%%%%%%%%%%%%%%%%%%%%%%%%%%%%%%%%%%
%%%%%%%%%%%%%%%%%%%%%%%%%%%%  Musterl�sung-Logo %%%%%%%%%%%%%%%%%%%%%%%%%%%
%%%%%%%%%%%%%%%%%%%%%%%%%%%%%%%%%%%%%%%%%%%%%%%%%%%%%%%%%%%%%%%%%%%%%%%%%%%
\newcommand{\mlogo}{ \sffamily \vspace*{-10mm}
  \begin{minipage}{2cm}
    \includegraphics[scale=0.3]{Bilder/KIT.jpg}
  \end{minipage}
  \hfill
  \begin{minipage}{10.5cm}
    %\textsf{{\Large U}\large{NIVERSIT"AT} \Large{K}\large{ARLSRUHE}} \\
    \textsf{\large{Karlsruhe Institute of Technology}} \\[0.2ex]
    \textsf{\large{Dr. Eunah Cho}}\\  
    \textsf{\large{Institute for Anthropomatics and Robotics, Interactive Systems Labs}}\\[0.2ex]  
    \textsf{eunah.cho@kit.edu} 
  \end{minipage}
  \rmfamily
  \vspace*{2cm}

%   \centerline{\Huge Tutorial}
   \centerline{\Large \textbf{MT Praktikum - Evaluation}  } 
   \vspace{1ex}
   \centerline{\Large 13. Feb 2017}

% 
%   \begin{minipage}{2cm}
%     \includegraphics[scale=0.3]{Bilder/KIT.jpg}
%   \end{minipage}
%   \hfill
%   \begin{minipage}{10.5cm}
%     %\textsf{{\Large U}\large{NIVERSIT"AT} \Large{K}\large{ARLSRUHE}} \\
%     \textsf{\large{Fakult"at f"ur Informatik}} \\[0.2ex]
%     \textsf{\large{Prof. Dr. A. Waibel}}\\  
%     \textsf{\large{Prof. Dr. R. Dillmann}}\\[0.2ex]  
%     \textsf{Institut f�r Anthropomatik (IFA)}    
%   \end{minipage}
%   \rmfamily
%   \vspace*{2cm}
% 
%   \centerline{\Huge Tutorial}
%   \centerline{\Large CRF  } 
%   \vspace{1ex}
%   \centerline{\Large am 18. September 2012,  11:00 - 12:00 Uhr}
%   \begin{table}[htb]
%     \begin{center}
%       \leavevmode
%       \begin{tabular}[t]{|p{5cm}|p{5cm}|p{5cm}|} \hline
%         Name: & Vorname: & Matrikelnummer:\\[1ex]
%         \textsf{\Large Burns} & \textsf{\Large Charles Montgomery} & \textsf{\Large 0000002}\\[1ex] \hline
%       \end{tabular}
%     \end{center}
%   \end{table}
%   \vspace*{-1cm}
%   \renewcommand{\arraystretch}{1.3}
%   \begin{table}[!hb]
%     \begin{center}
%       \leavevmode
%       \begin{tabular}[t]{|l|rl|} \hline
%         %%\multicolumn{3}{|l|}{ }\\ 
%         Aufgabe 1 \hspace{5.5cm} & \hspace{3.5cm} & 10 von 10 Punkten\\
%         \hline
%         Aufgabe 2 \hspace{5.5cm} & \hspace{3.5cm} & 10 von 10
%         Punkten\\
%         \hline 
%         Aufgabe 3 \hspace{5.5cm} & \hspace{3.5cm} & 10 von 10
%         Punkten\\
%         \hline
%         Aufgabe 4 \hspace{5.5cm} & \hspace{3.5cm} & 10 von 10
%         Punkten\\
%         \hline 
%         Aufgabe 5 \hspace{5.5cm} & \hspace{3.5cm} & 10 von 10
%         Punkten\\
%         \hline
%         Aufgabe 6 \hspace{5.5cm} & \hspace{3.5cm} & 10 von 10
%         Punkten\\
%         \hline
%         Online~~~ \hspace{5.5cm} & \hspace{3.5cm} & 6 von 6 Punkten\\ \hline  
%         \hline
%         \multicolumn{3}{c}{ }\\ \hline
%         \textbf{Gesamtpunktzahl:} &  & 66 von 66 Punkten \\ \hline
%         \multicolumn{3}{c}{ } \\ \hline
%         & \multicolumn{2}{|l|}{\textbf{Note:\hspace*{2.5cm} \textsf{1,0}}}\\\hline
%       \end{tabular}
%     \end{center}
%   \end{table}
%   \newpage
  } %end M-Logo
%%%%%%%%%%%%%%%%%%%%%%%%% Headings (Musterl�sung) %%%%%%%%%%%%%%%%%%%%%
\newcommand{\mhead}{\pagestyle{myheadings}
        \markright{\underline{\textsf{Tutorial - 13.Feb.2017  \hspace{4cm} Eunah Cho, KIT, eunah.cho@kit.edu}}}}
%%%%%%%%%%%%%%%%%%%%%%%%%%%%%%%%%%%%%%%%%%%%%%%%%%%%%%%%%%%%%%%%%%%%%%%%

\newcommand{\pkt}[1]{\marginpar{\fbox{\textsf{#1 P.}}}}
\def\Not#1{\overline{\strut #1}}
\newcommand{\rb}[1]{\raisebox{1.5ex}[-1.5ex]{#1}}
\newcommand{\rbp}[1]{\raisebox{-1.5ex}[1.5ex]{#1}}
\newcommand{\rbg}[1]{\raisebox{2.5ex}[-2.5ex]{#1}}

%%%%%%%%%%%%%%%%%%%%%%%% TI-spezifische Definitionen %%%%%%%%%%%%%%%%%%%%
\newcommand{\mikpro}{$\mu P$ }
\newcommand{\aktivlow}[1]{$\overline{\mbox{{\sf{#1}}}}$}
\newcommand{\lowsignal}[1]{$\overline{\mbox{{#1}}}$}
%
\newcommand{\mytext}[2]{$\mbox{#1}_{\mbox{#2}}$}
\newcommand{\bus}[4]{$\mbox{#1}_{\mbox{#2}}-\mbox{#3}_{\mbox{#4}}$}
\newcommand{\notvar}[1]{$\overline{\mbox{#1}}$}
%
\newcommand{\antii}{\leftrightarrow \hspace{-2.2ex}  | \hspace{1.2ex}}
\newcommand{\anti}{\not\leftrightarrow}
\newcommand{\aqui}{\leftrightarrow}
%
\newcommand{\oa}{\overline{a}}
\newcommand{\ob}{\overline{b}}
\newcommand{\oc}{\overline{c}}
\newcommand{\od}{\overline{d}}
\newcommand{\oee}{\overline{e}}
\newcommand{\ox}{\overline{x}}
\newcommand{\oy}{\overline{y}}
\newcommand{\oz}{\overline{z}}
\newcommand{\ow}{\overline{w}}
\newcommand{\oq}{\overline{q}}
\newcommand{\ua}{\underline{a}}
\newcommand{\ub}{\underline{b}}
%
\newcommand{\oben}[1]{\overline{#1}}
\newcommand{\unten}[1]{\underline{#1}}
%
\newcommand{\nand}[1]{\mbox{NAND}_{#1}}
\newcommand{\nor}[1]{\mbox{NOR}_{#1}}
%
\renewcommand{\figurename}{Bild}

%%%%%%%%%%%%%%%%%%%%%%%%%%%%%%%%%%%%%%%%%%%%%%%%%%%%%%%%%%%%%%%%%%%%%%%%
%%%%%%%%%%%%%%%%%%%%%%%%%%%% PS mit LaTeX  %%%%%%%%%%%%%%%%%%%%%%%%%%%%%
\newenvironment{Abbildung}[2]{\edef\MyCaption{#1}\edef\MyRef{#2}%
  \begin{figure}[h!]\begin{center}}{%
      \caption{\MyCaption}\label{\MyRef}\end{center}\end{figure}}
% EpsAbbildung
\newcommand{\EpsAbbildung}[4]{%
  \begin{Abbildung}{#2}{#3}%
    \includegraphics[#4]{#1}%
  \end{Abbildung}}
% PsTexAbbildung 
\newcommand{\PsTexAbbildung}[3]{%
  \begin{Abbildung}{#2}{#3}%
    \begin{picture}(0,0)%
      \includegraphics{#1.pstex}%
    \end{picture}%
    \input #1.pstex_t%
  \end{Abbildung}}
%%%%%%%%%%%%%%%%%%%%%%%%%
\newenvironment{AbbildungOhneCaption}{
  \begin{figure}[h!]\begin{center}}{%
    \end{center}\end{figure}}%
%
\newcommand{\EpsAbbildungOhneCaption}[2]{%
  \begin{AbbildungOhneCaption}%
    \includegraphics[#2]{#1}%
  \end{AbbildungOhneCaption}}
% 
\newcommand{\PsTexAbbildungOhneCaption}[1]{%
  \begin{AbbildungOhneCaption}%
    \begin{picture}(0,0)%
      \includegraphics{#1.pstex}%
    \end{picture}%
    \input #1.pstex_t%
  \end{AbbildungOhneCaption}}%
%%%%%%%%%%%%%%%%%%%%%%%%%
\newcommand{\EpsAbbildungOhneCenter}[1]{%
  \begin{figure}%
    \includegraphics{#1}%
  \end{figure}}
% 
\newcommand{\PsTexAbbildungOhneCenter}[1]{%
  \begin{picture}(0,0)%
    \includegraphics{#1.pstex}%
  \end{picture}%
  \input #1.pstex_t%
  }

%%% Trennungsliste
\hyphenation{Prim-implikant Prim-implikat Kern-prim-implikant
  Kern-prim-implikat Distributiv-gesetz Hasard-fehler Pipe-lining}

%%% Local Variables: 
%%% mode: latex
%%% TeX-master: t
%%% End: 







